\documentclass{ctexart}
% 此处引入常用包,从此行到46行均无需修改
\usepackage[dvipsnames, svgnames, x11names]{xcolor}
\usepackage{listings}
\usepackage{graphicx}
\usepackage{tabularx}
\usepackage[most]{tcolorbox}
\usepackage{amsmath}
\usepackage{multicol}
\usepackage{multirow}
\usepackage{pifont}
\usepackage{enumitem}
\usepackage{bbding}
\usepackage{colortbl}
\usepackage{placeins}
\usepackage{mathpazo}
\usepackage{bm}
\usepackage{tikz}
\usepackage{xparse}
\usepackage{fancyhdr}


%定义题目计数器和命令
\newcounter{questioncnt}
\newcounter{subquestioncnt}[questioncnt]
\newcounter{subsubquestioncnt}[subquestioncnt]

\NewDocumentCommand\question{om}{\noindent\IfNoValueTF{#1}{\textcolor{blue}{\stepcounter{questioncnt}\arabic{questioncnt}}}{#1}\quad#2\par}
\NewDocumentCommand\subquestion{om}{\noindent\IfNoValueTF{#1}{\textcolor{blue}{\stepcounter{subquestioncnt}\arabic{questioncnt}.\arabic{subquestioncnt}}}{#1}\quad#2\par}
\NewDocumentCommand\subsubquestion{om}{\noindent\IfNoValueTF{#1}{\textcolor{blue}{\stepcounter{subsubquestioncnt}\arabic{questioncnt}.\arabic{subquestioncnt}.\arabic{subsubquestioncnt}}}{#1}\quad#2\par}

%定义回答计数器和命令
\newcounter{answercnt}
\newcounter{subanswercnt}[answercnt]
\newcounter{subsubanswercnt}[subanswercnt]

\NewDocumentCommand\answer{o}{\noindent\textcolor{blue}{\IfNoValueTF{#1}{\stepcounter{answercnt}\arabic{answercnt}}{#1}}\quad}
\NewDocumentCommand\subanswer{o}{\noindent\textcolor{blue}{\IfNoValueTF{#1}{\stepcounter{subanswercnt}\arabic{answercnt}.\arabic{subanswercnt}}{#1}}\quad}
\NewDocumentCommand\subsubanswer{o}{\noindent\textcolor{blue}{\IfNoValueTF{#1}{\stepcounter{subsubanswercnt}\arabic{answercnt}.\arabic{subanswercnt}.\arabic{subsubanswercnt}}{#1}}\quad}

%在此处进行基本信息修改
\newcommand{\sCourse}{计算机系统结构}   %课程名
\newcommand{\nTime}{2}             %作业次数
\newcommand{\sName}{黄昊}           %学生姓名
\newcommand{\sNumber}{20204205}     %学号

%页边距设置
\usepackage[left=2cm,right=2cm,top=3cm,bottom=2cm]{geometry}

%页眉页脚设置
\pagestyle{fancy}
\fancyhead[C]{\today}

\newcommand{\homeworkTitle}{
    \setcounter{answercnt}{0}
    %标题部分修改
    \begin{center}
        \fontsize{16pt}{0}{\textbf{\kaishu\sCourse课程\quad第\nTime次作业}}\\
        \fontsize{13pt}{0}{\textit{\kaishu\sName\qquad\sNumber}}\\
    \end{center}}

\begin{document}
\homeworkTitle
\begin{itemize}
    \item {Consider the unpipelined processor. Assume 
        that it has a \textit{1ns} clock cycle and that it 
        uses 
        \textit{4 cycles }
        for branches and stores and 
        \textit{5 cycles} for other operations.
        Assume that the relative frequencies of branch and 
        store operations are \textit{15\% and 10\%}, 
        respectively. }
    \item {Consider a pipelined processor.
        Assume the slowest stage takes \textit{1ns} and clock
        skew and register setup add \textit{0.2 ns} to the clock 
        period. }
    \item {\textbf{How much speedup in the instruction 
        execution rate will we gain from an ideal pipeline?}}
\end{itemize}
    
    \answer[\textbf{Solution:}] 
    
    execution time for \textbf{pipelined processor}:
    $1ns+0.2ns=1.2ns$

    execution time for \textbf{unpipelined processor}:
    $(15\%+10\%)\times 4+(1-25\%)\times 5=4.75ns$

    \textbf{speedup}: $4.75ns/1.2ns=3.96$
\end{document}