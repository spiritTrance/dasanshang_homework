\documentclass{ctexart}
% 此处引入常用包,从此行到46行均无需修改
\usepackage[dvipsnames, svgnames, x11names]{xcolor}
\usepackage{listings}
\usepackage{graphicx}
\usepackage{tabularx}
\usepackage[most]{tcolorbox}
\usepackage{amsmath}
\usepackage{multicol}
\usepackage{multirow}
\usepackage{pifont}
\usepackage{enumitem}
\usepackage{bbding}
\usepackage{colortbl}
\usepackage{placeins}
\usepackage{mathpazo}
\usepackage{bm}
\usepackage{tikz}
\usepackage{xparse}
\usepackage{fancyhdr}
\usepackage[ruled,linesnumbered]{algorithm2e}
\usepackage{algorithmic}

%代码环境设置
\lstset{
    numbers = none ,                                    %可选参数有none,right,left
    breaklines ,                                        %换行有影响,不加这个则换行时从头开始
    numberstyle = \tiny ,                               %数字大小’
    keywordstyle = \color{blue!70} ,                    %关键字颜色
    commentstyle =\color{black!40!white} ,              %注释颜色
    frame = shadowbox ,                                 %阴影设置
    rulesepcolor = \color{red!20!green!20!blue!20} ,    %阴影颜色设置
    escapeinside =`',                                   %lst中文支持不太好,可以用这个括在中文旁边
    basicstyle =\footnotesize\ttfamily                  %代码字体设置
}

%定义题目计数器和命令
\newcounter{questioncnt}
\newcounter{subquestioncnt}[questioncnt]
\newcounter{subsubquestioncnt}[subquestioncnt]

\NewDocumentCommand\question{om}{\noindent\IfNoValueTF{#1}{\textcolor{blue}{\stepcounter{questioncnt}\arabic{questioncnt}}}{\textcolor{blue}{#1}}\quad#2\par}
\NewDocumentCommand\subquestion{om}{\noindent\IfNoValueTF{#1}{\textcolor{blue}{\stepcounter{subquestioncnt}\arabic{questioncnt}.\arabic{subquestioncnt}}}{#1}\quad#2\par}
\NewDocumentCommand\subsubquestion{om}{\noindent\IfNoValueTF{#1}{\textcolor{blue}{\stepcounter{subsubquestioncnt}\arabic{questioncnt}.\arabic{subquestioncnt}.\arabic{subsubquestioncnt}}}{#1}\quad#2\par}

%定义回答计数器和命令
\newcounter{answercnt}
\newcounter{subanswercnt}[answercnt]
\newcounter{subsubanswercnt}[subanswercnt]

\NewDocumentCommand\answer{o}{\noindent\textcolor{blue}{\IfNoValueTF{#1}{\stepcounter{answercnt}\arabic{answercnt}}{#1}}\quad}
\NewDocumentCommand\subanswer{o}{\noindent\textcolor{blue}{\IfNoValueTF{#1}{\stepcounter{subanswercnt}\arabic{answercnt}.\arabic{subanswercnt}}{#1}}\quad}
\NewDocumentCommand\subsubanswer{o}{\noindent\textcolor{blue}{\IfNoValueTF{#1}{\stepcounter{subsubanswercnt}\arabic{answercnt}.\arabic{subanswercnt}.\arabic{subsubanswercnt}}{#1}}\quad}

%在此处进行基本信息修改
\newcommand{\sCourse}{计算机系统结构}   %课程名
\newcommand{\nTime}{4}             %作业次数
\newcommand{\sName}{黄昊}           %学生姓名
\newcommand{\sNumber}{20204205}     %学号

%页边距设置
\usepackage[left=2cm,right=2cm,top=3cm,bottom=2cm]{geometry}

%页眉页脚设置
\pagestyle{fancy}
\fancyhead[C]{\today}

\newcommand{\homeworkTitle}{
    \setcounter{answercnt}{0}
    %标题部分修改
    \begin{center}
        \fontsize{16pt}{0}{\textbf{\kaishu\sCourse课程\quad第\nTime次作业}}\\
        \fontsize{13pt}{0}{\textit{\kaishu\sName\qquad\sNumber}}\\
    \end{center}}

\begin{document}
    \setcounter{answercnt}{0}
    %标题部分修改
    \begin{center}
        \fontsize{16pt}{0}{\textbf{\kaishu\sCourse课程\quad第\nTime次作业}}\\
        \fontsize{13pt}{0}{\textit{\kaishu\sName\qquad\sNumber}}
    \end{center}


    \noindent Given:  A cache can hold 64 KByte data. Data are transferred between main memory and the cache in blocks of 4 bytes each. The main memory consists of 16 Mbytes, with each byte directly accessible by a 24-bit address.

    
    
    \question[A]{How many cache blocks in the cache?} 
    $$
    the\ number\ of\ cache\ block\ in\ the\ cache = \frac{64KB}{4B} = 16K = 2^{14} 
    $$


    
    
    \question[B]{How many blocks in the memory?}

    $$
    the\ number\ of\ cache\ block\ in\ the\ cache = \frac{16MB}{4B} = 4M = 2^{22} 
    $$


    
    
    \question[C]{For the direct-mapping cache, how many bits for tag and cache line in a memory address, respectively?}
    $$
    \begin{aligned}
        bits\ for\ tag &=24 - 14 - 2 = 8bits\\
        bits\ for\ cache\ line &= 4\times 8+8+1= 41bits
    \end{aligned}
    $$


    
    
    \question[D]{For the associative cache, how many bits for tag in a memory address?}
    $$
        bits\ for\ tag =24 - 2 = 22
    $$


    
    
    \question[E]{For the 4-way set associative cache, how many bits for tag and set\footnote{Due to the ambiguous of the question, the bits for set and the number of set was both answered.} in a memory address, respectively?}
    $$
    \begin{aligned}
        bits\ for\ tag &=24 - 12 - 2 = 10\\
        bits\ for\ set &=4\times 8\times 4 = 128bits\\
        numbers\ for\ set &=\frac{64KB}{4\times 4B}=4K=2^{12}
    \end{aligned}
    $$


    
    
    \question[F]{How many tag comparisons are needed in each memory access for direct-mapping, associative and four-way set associate cache, respectively?}
    $$
    \begin{aligned}
        the\ number\ of\ comparisons\ in\ direct\ mapping&:1\\
        the\ number\ of\ comparisons\ in\ associative&:2^{14}\\
        the\ number\ of\ comparisons\ in\ four\ way\ set\ associate&:4\\
    \end{aligned}
    $$


\end{document}
