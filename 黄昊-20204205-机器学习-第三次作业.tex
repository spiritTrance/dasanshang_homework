\documentclass{ctexart}
% 此处引入常用包,从此行到46行均无需修改
\usepackage[dvipsnames, svgnames, x11names]{xcolor}
\usepackage{listings}
\usepackage{graphicx}
\usepackage{tabularx}
\usepackage[most]{tcolorbox}
\usepackage{amsmath}
\usepackage{multicol}
\usepackage{multirow}
\usepackage{pifont}
\usepackage{enumitem}
\usepackage{bbding}
\usepackage{colortbl}
\usepackage{placeins}
\usepackage{mathpazo}
\usepackage{bm}
\usepackage{tikz}
\usepackage{xparse}
\usepackage{fancyhdr}
\usepackage[ruled,linesnumbered]{algorithm2e}
\usepackage{algorithmic}


%定义题目计数器和命令
\newcounter{questioncnt}
\newcounter{subquestioncnt}[questioncnt]
\newcounter{subsubquestioncnt}[subquestioncnt]

\NewDocumentCommand\question{om}{\noindent\IfNoValueTF{#1}{\textcolor{blue}{\stepcounter{questioncnt}\arabic{questioncnt}}}{#1}\quad#2\par}
\NewDocumentCommand\subquestion{om}{\noindent\IfNoValueTF{#1}{\textcolor{blue}{\stepcounter{subquestioncnt}\arabic{questioncnt}.\arabic{subquestioncnt}}}{#1}\quad#2\par}
\NewDocumentCommand\subsubquestion{om}{\noindent\IfNoValueTF{#1}{\textcolor{blue}{\stepcounter{subsubquestioncnt}\arabic{questioncnt}.\arabic{subquestioncnt}.\arabic{subsubquestioncnt}}}{#1}\quad#2\par}

%定义回答计数器和命令
\newcounter{answercnt}
\newcounter{subanswercnt}[answercnt]
\newcounter{subsubanswercnt}[subanswercnt]

\NewDocumentCommand\answer{o}{\noindent\textcolor{blue}{\IfNoValueTF{#1}{\stepcounter{answercnt}\arabic{answercnt}}{#1}}\quad}
\NewDocumentCommand\subanswer{o}{\noindent\textcolor{blue}{\IfNoValueTF{#1}{\stepcounter{subanswercnt}\arabic{answercnt}.\arabic{subanswercnt}}{#1}}\quad}
\NewDocumentCommand\subsubanswer{o}{\noindent\textcolor{blue}{\IfNoValueTF{#1}{\stepcounter{subsubanswercnt}\arabic{answercnt}.\arabic{subanswercnt}.\arabic{subsubanswercnt}}{#1}}\quad}

%在此处进行基本信息修改
\newcommand{\sCourse}{机器学习}   %课程名
\newcommand{\nTime}{4}             %作业次数
\newcommand{\sName}{黄昊}           %学生姓名
\newcommand{\sNumber}{20204205}     %学号

%页边距设置
\usepackage[left=2cm,right=2cm,top=3cm,bottom=2cm]{geometry}

%页眉页脚设置
\pagestyle{fancy}
\fancyhead[C]{\today}

\newcommand{\homeworkTitle}{
    \setcounter{answercnt}{0}
    %标题部分修改
    \begin{center}
        \fontsize{16pt}{0}{\textbf{\kaishu\sCourse课程\quad第\nTime次作业}}\\
        \fontsize{13pt}{0}{\textit{\kaishu\sName\qquad\sNumber}}\\
    \end{center}}

\begin{document}
\homeworkTitle
\answer[4.1]
显然成立:构造这样一颗决策树:第一层判断特征向量的第一个分量,第二层判断第二个...
以此类推。由于数据各不相同,故这样构造出来1决策树,必然能分到一个叶节点,且只有一
个数据符合。根据这个构造方法,每个数据到达叶节点的路径各不相同,且一定完全符合(因
为各不冲突),故训练误差为0.

\answer[4.2]
把训练误差作为训练准则容易出现泛化能力差的问题。

\answer[4.3]


\answer[4.4]


\answer[4.8]
算法见下页。
\begin{algorithm}
\caption{决策树生成算法——基于广度优先搜索}\label{algorithm}
\KwData{训练集D=\{$(x_1,y_1),(x_2,y_2),\dots,(x_m,y_m)$\}\\\qquad\ \ \ 属性集A=\{$a_1,a_2,\dots,a_d$\}\\\qquad\ \ \ 最大高度 MaxDepth}
\KwResult{决策树$T$}
生成节点$N$,节点信息包括数据集$D$,属性集$A$,高度信息$h$;\\
记录决策树T的根为N;\\
生成节点队列Q;\\
将N压入队列Q的队尾;\\
\While{节点队列Q非空}{
    从节点队列Q中取出队首节点N;\\
    \If{节点N.D中样本全属于同一类别C}{将N标记为C类叶节点;\textbf{continue};}
    \If{节点N.h已达到MaxDepth \textbf{OR} $N.A=\emptyset$ \textbf{OR} N.D中样本在N.A上的取值相同}{将N标记为叶节点,其类别标记为N.D中样本最多的类;\textbf{continue};}
    从N.A中选择最优划分属性$a_*$;\\
    \For{$a_*$的每一个值$a_*^v$}{
        为N生成一个分支;令$D_v$为D中在$a_*$上取值为$a_*^v$的样本子集;\\
        \eIf{$D_v$为空}
        {将分支节点标记为叶节点,其类别表及为D中样本最多的类;continue;}
        {生成节点$N_s$,节点信息包括数据集$D_v$,属性集$A$\textbackslash$\{a_*\}$,高度信息$N.h+1$;\\将$N_s$压入节点队列Q}
    }
}
\textbf{return} 决策树T
\end{algorithm}
如果属性取值较多但属性少,BFS比DFS空间消耗更大;若属性多但属性值少,则DFS比BFS空间消耗更大,DFS有爆栈的风险。
\end{document}