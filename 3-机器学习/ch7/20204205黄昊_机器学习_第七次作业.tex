\documentclass{ctexart}
% 此处引入常用包,从此行到46行均无需修改
\usepackage[dvipsnames, svgnames, x11names]{xcolor}
\usepackage{listings}
\usepackage{graphicx}
\usepackage{tabularx}
\usepackage[most]{tcolorbox}
\usepackage{amsmath}
\usepackage{multicol}
\usepackage{multirow}
\usepackage{pifont}
\usepackage{enumitem}
\usepackage{bbding}
\usepackage{colortbl}
\usepackage{placeins}
\usepackage{mathpazo}
\usepackage{bm}
\usepackage{tikz}
\usepackage{xparse}
\usepackage{fancyhdr}
\usepackage[ruled,linesnumbered]{algorithm2e}
\usepackage{algorithmic}

%代码环境设置
\lstset{
    numbers = none ,                                    %可选参数有none,right,left
    breaklines ,                                        %换行有影响,不加这个则换行时从头开始
    numberstyle = \tiny ,                               %数字大小’
    keywordstyle = \color{blue!70} ,                    %关键字颜色
    commentstyle =\color{black!40!white} ,              %注释颜色
    frame = shadowbox ,                                 %阴影设置
    rulesepcolor = \color{red!20!green!20!blue!20} ,    %阴影颜色设置
    escapeinside =`',                                   %lst中文支持不太好,可以用这个括在中文旁边
    basicstyle =\footnotesize\ttfamily                  %代码字体设置
}

%定义题目计数器和命令
\newcounter{questioncnt}
\newcounter{subquestioncnt}[questioncnt]
\newcounter{subsubquestioncnt}[subquestioncnt]

\NewDocumentCommand\question{om}{\noindent\IfNoValueTF{#1}{\textcolor{blue}{\stepcounter{questioncnt}\arabic{questioncnt}}}{#1}\quad#2\par}
\NewDocumentCommand\subquestion{om}{\noindent\IfNoValueTF{#1}{\textcolor{blue}{\stepcounter{subquestioncnt}\arabic{questioncnt}.\arabic{subquestioncnt}}}{#1}\quad#2\par}
\NewDocumentCommand\subsubquestion{om}{\noindent\IfNoValueTF{#1}{\textcolor{blue}{\stepcounter{subsubquestioncnt}\arabic{questioncnt}.\arabic{subquestioncnt}.\arabic{subsubquestioncnt}}}{#1}\quad#2\par}

%定义回答计数器和命令
\newcounter{answercnt}
\newcounter{subanswercnt}[answercnt]
\newcounter{subsubanswercnt}[subanswercnt]

\NewDocumentCommand\answer{o}{\noindent\textcolor{blue}{\IfNoValueTF{#1}{\stepcounter{answercnt}\arabic{answercnt}}{#1}}\quad}
\NewDocumentCommand\subanswer{o}{\noindent\textcolor{blue}{\IfNoValueTF{#1}{\stepcounter{subanswercnt}\arabic{answercnt}.\arabic{subanswercnt}}{#1}}\quad}
\NewDocumentCommand\subsubanswer{o}{\noindent\textcolor{blue}{\IfNoValueTF{#1}{\stepcounter{subsubanswercnt}\arabic{answercnt}.\arabic{subanswercnt}.\arabic{subsubanswercnt}}{#1}}\quad}

%在此处进行基本信息修改
\newcommand{\sCourse}{机器学习}   %课程名
\newcommand{\nTime}{7}             %作业次数
\newcommand{\sName}{黄昊}           %学生姓名
\newcommand{\sNumber}{20204205}     %学号

%页边距设置
\usepackage[left=2cm,right=2cm,top=3cm,bottom=2cm]{geometry}

%页眉页脚设置
\pagestyle{fancy}
\fancyhead[C]{\today}

\newcommand{\homeworkTitle}{
    \setcounter{answercnt}{0}
    %标题部分修改
    \begin{center}
        \fontsize{16pt}{0}{\textbf{\kaishu\sCourse课程\quad第\nTime次作业}}\\
        \fontsize{13pt}{0}{\textit{\kaishu\sName\qquad\sNumber}}\\
    \end{center}}

\begin{document}
    \setcounter{answercnt}{0}
    %标题部分修改
    \begin{center}
        \fontsize{16pt}{0}{\textbf{\kaishu\sCourse课程\quad第\nTime次作业}}\\
        \fontsize{13pt}{0}{\textit{\kaishu\sName\qquad\sNumber}}
    \end{center}

    \noindent选择题目\answer[7.1]\answer[7.4]\answer[7.5]\answer[7.7]\answer[7.8]
    

    \answer[7.1]
\begin{itemize}
    \item 色泽
    $$
    \begin{aligned}
        P(\mbox{青绿}|\mbox{好瓜})=\frac{3}{8}\\
        P(\mbox{乌黑}|\mbox{好瓜})=\frac{1}{2}\\
        P(\mbox{浅白}|\mbox{好瓜})=\frac{1}{8}\\
        P(\mbox{青绿}|\mbox{坏瓜})=\frac{1}{3}\\
        P(\mbox{乌黑}|\mbox{坏瓜})=\frac{2}{9}\\
        P(\mbox{浅白}|\mbox{坏瓜})=\frac{4}{9}\\
    \end{aligned}
    $$
    \item 根蒂
    $$
    \begin{aligned}
    P(\mbox{蜷缩}|\mbox{好瓜})=\frac{5}{8}\\
    P(\mbox{稍蜷}|\mbox{好瓜})=\frac{3}{8}\\
    P(\mbox{硬挺}|\mbox{好瓜})={0}\\
    P(\mbox{蜷缩}|\mbox{坏瓜})=\frac{1}{3}\\
    P(\mbox{稍蜷}|\mbox{坏瓜})=\frac{4}{9}\\
    P(\mbox{硬挺}|\mbox{坏瓜})=\frac{2}{9}\\
    \end{aligned}
    $$
    \item 敲声
    $$
    \begin{aligned}
    P(\mbox{浊响}|\mbox{好瓜})=\frac{3}{4}\\
    P(\mbox{沉闷}|\mbox{好瓜})=\frac{1}{4}\\
    P(\mbox{清脆}|\mbox{好瓜})={0}\\
    P(\mbox{浊响}|\mbox{坏瓜})=\frac{4}{9}\\
    P(\mbox{沉闷}|\mbox{坏瓜})=\frac{1}{3}\\
    P(\mbox{清脆}|\mbox{坏瓜})=\frac{2}{9}\\
    \end{aligned}
    $$
\end{itemize}
    

    \answer[7.4]
概率取对数,连乘转化成连加即可。
    

    \answer[7.5]
    对于最小化分类错误率的贝叶斯最优分类器,有:
    $$
    \begin{aligned}
        h^*(x) &= arg\max_{c\in y}P(c|x)\\
        &=arg\max_{c\in y}P(x|c)P(c)\\
        &= arg\max_{c\in y}\frac{1}{(2\pi)^{n/2}|\Sigma|^{1/2}}\exp(-\frac{1}{2}(x-\mu_c)^T\Sigma^{-1}(x-\mu_c))P(c)\\
        &= arg\max_{c\in y}\ln(\frac{1}{(2\pi)^{n/2}|\Sigma|^{1/2}}\exp(-\frac{1}{2}(x-\mu_c)^T\Sigma^{-1}(x-\mu_c))P(c))\\
        &= arg\max_{c\in y}-\frac{1}{2}(x-\mu_c)^T\Sigma^{-1}(x-\mu_c)+\ln P(c)\\
        &= arg\max_{c\in y}x^T\Sigma^{-1}\mu_c-\frac{1}{2}\mu_c^T\Sigma^{-1}\mu_c+\ln P(c)
    \end{aligned}
    $$

    对于二分类问题,贝叶斯决策边界可以表示为:
    $$
    g(x) = x^T\Sigma^{-1}(\mu_1-\mu_0)-\frac{1}{2}(\mu_1+\mu_0)^T\Sigma^{-1}(\mu_1-\mu_0)+\ln(\frac{P(1)}{P(0)})
    $$

    对于线性判别分析,投影界面$w=(\Sigma_0+\Sigma_1)^{-1}(\mu_1-\mu_0)$,两类别在投影面连线中点为:$w=\frac{1}{2}\Sigma^{-1}(\mu_1-\mu_0)w$,则LDA的决策边界为$g(x)=x^T\Sigma^{-1}(\mu_1-\mu_0)-\frac{1}{2}(\mu_1+\mu_0)^T\Sigma^{-1}(\mu_1-\mu_0)$,此时仅相差$\ln\frac{P(1)}{P(0)}$项,由贝叶斯学派的同等无知原则,令$P(1)=P(0)$,即可消去此项,从而得证。


    \answer[7.7]
规定问题为\textbf{二分类}问题,样本有$d$个属性,在第$i$个属性上的取值种类个数为$n_i$个,则最好的情况为:每个样本有$d$个属性,每个属性都满足30个的最低要求,
则最少个数为$2\times 30\times\max_{1\leq i\leq d}\{n_i\}= 60\max_{1\leq i\leq d}\{n_i\}$
个。
最坏的情况为:从第一个属性开始,选取的样本满足要求后,在第二个属性上的取
值完全相同,第三个及之后的属性以此类推。
不难发现最坏的样本个数为:
$$
\begin{aligned}
    & \qquad 2\times 30\times n_1 + 2\times 30\times(n_2 - 1) + 2\times 30\times (n_3 - 1) + \dots + 2\times 30 \times (n_d - 1)\\
    & = 60\sum_{i=1}^d(n_i - 1) +60\\
    & = 60\sum_{i=1}^dn_i + 60(1-d)
\end{aligned}
$$

    \answer[7.8]
    对于同父结构:
    $$
\begin{aligned}
    P(x_3,x_4) & = \sum_{x_1}P(x_1,x_3,x_4)\\
               & = \sum_{x_1}P(x_1)P(x_3|x_1)P(x_4|x_1)\\
               & \neq P(x_3)P(x_4)    
\end{aligned}
$$
因此$x_3,x_4$关于$x_1$边际独立不成立。

对于顺序结构:
$$
    P(x, y, z) = P(z) P(x|z) P(y|x)
$$

给定x时:
$$
\begin{aligned}
    P(y, z|x) &= \frac{P(z)P(x|z)P(y|x)}{P(x)} \\
              &= \frac{P(x, z)P(y|x)}{P(x)}\\
              &= {P(z|x)P(y|x)}
\end{aligned}
$$
即顺序结构中$y,z$关于$x$条件独立。

$x$取值未知时:
$$
\begin{aligned}
    P(y,z) &= \sum_x P(x,y,z)\\
    &= \sum_x P(z)P(x|z)P(y|x)\\
    & \neq P(y)P(z)
\end{aligned}
$$
所以$y,z$关于$x$边际独立不成立。  
\end{document}
