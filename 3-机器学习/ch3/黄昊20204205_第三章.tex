\documentclass{ctexart}
% 此处引入常用包,从此行到46行均无需修改
\usepackage[dvipsnames, svgnames, x11names]{xcolor}
\usepackage{listings}
\usepackage{graphicx}
\usepackage{tabularx}
\usepackage[most]{tcolorbox}
\usepackage{amsmath}
\usepackage{multicol}
\usepackage{multirow}
\usepackage{pifont}
\usepackage{enumitem}
\usepackage{bbding}
\usepackage{colortbl}
\usepackage{placeins}
\usepackage{mathpazo}
\usepackage{bm}
\usepackage{tikz}
\usepackage{xparse}
\usepackage{fancyhdr}


%定义题目计数器和命令
\newcounter{questioncnt}
\newcounter{subquestioncnt}[questioncnt]
\newcounter{subsubquestioncnt}[subquestioncnt]

\NewDocumentCommand\question{om}{\noindent\IfNoValueTF{#1}{\textcolor{blue}{\stepcounter{questioncnt}\arabic{questioncnt}}}{#1}\quad#2\par}
\NewDocumentCommand\subquestion{om}{\noindent\IfNoValueTF{#1}{\textcolor{blue}{\stepcounter{subquestioncnt}\arabic{questioncnt}.\arabic{subquestioncnt}}}{#1}\quad#2\par}
\NewDocumentCommand\subsubquestion{om}{\noindent\IfNoValueTF{#1}{\textcolor{blue}{\stepcounter{subsubquestioncnt}\arabic{questioncnt}.\arabic{subquestioncnt}.\arabic{subsubquestioncnt}}}{#1}\quad#2\par}

%定义回答计数器和命令
\newcounter{answercnt}
\newcounter{subanswercnt}[answercnt]
\newcounter{subsubanswercnt}[subanswercnt]

\NewDocumentCommand\answer{o}{\noindent\textcolor{blue}{\IfNoValueTF{#1}{\stepcounter{answercnt}\arabic{answercnt}}{#1}}\quad}
\NewDocumentCommand\subanswer{o}{\noindent\textcolor{blue}{\IfNoValueTF{#1}{\stepcounter{subanswercnt}\arabic{answercnt}.\arabic{subanswercnt}}{#1}}\quad}
\NewDocumentCommand\subsubanswer{o}{\noindent\textcolor{blue}{\IfNoValueTF{#1}{\stepcounter{subsubanswercnt}\arabic{answercnt}.\arabic{subanswercnt}.\arabic{subsubanswercnt}}{#1}}\quad}

%在此处进行基本信息修改
\newcommand{\sCourse}{机器学习}   %课程名
\newcommand{\nTime}{3}             %作业次数
\newcommand{\sName}{黄昊}           %学生姓名
\newcommand{\sNumber}{20204205}     %学号

%页边距设置
\usepackage[left=2cm,right=2cm,top=3cm,bottom=2cm]{geometry}

%页眉页脚设置
\pagestyle{fancy}
\fancyhead[C]{\today}

\newcommand{\homeworkTitle}{
    \setcounter{answercnt}{0}
    %标题部分修改
    \begin{center}
        \fontsize{16pt}{0}{\textbf{\kaishu\sCourse课程\quad第\nTime次作业}}\\
        \fontsize{13pt}{0}{\textit{\kaishu\sName\qquad\sNumber}}\\
    \end{center}}

\begin{document}
\homeworkTitle
选择\answer[3.1]\answer[3.2]\answer[3.6]\answer[3.7]\answer[3.9].

\answer[3.1]
b被确认是一个常数,与输入无关时。此时对线性模型进行变换:$\boldsymbol{y}
_{new}=\boldsymbol{y}_{old}-{b}=\omega \boldsymbol{x}$
后即可不考虑偏置项${b}$

\answer[3.2]
\begin{equation*}
\begin{aligned}    
y&=\frac{1}{1+e^{-\boldsymbol{w}^T\boldsymbol{x}+{b}}}\\
\frac{\partial y}{\partial \boldsymbol{w}}&=-(1+e^{-\boldsymbol{w}
^T\boldsymbol{x}+{b}})^{-2}·(e^{-\boldsymbol{w}^T\boldsymbol
{x}+{b}})·(-x)\\
&={x(\frac{1}{y}-1)}{y^2}\\
&=x(y-y^2)\\
\frac{\partial^2 y}{\partial w\partial w^T}&=xx^T(1-2y)(y-y^2)
\end{aligned}
\end{equation*}
由半正定矩阵定义易知$xx^T$是半正定的。又因为$(1-2y)(y-y^2)$显然在$y=1/2$时为分界
点,则Hessian矩阵不能在y的取值范围内一直是半正定的,从而目标函数是非凸的。

$$
\begin{aligned}    
    l(\beta)&=\sum_{i=1}^{m}(-y_i\beta^T\hat{x}_i+\ln(1+e^{\beta^T\hat{x}_i}))\\
    \frac{\partial l}{\partial \beta}&=\sum_{i=1}^m(-y_i\hat {x}_i+\frac{\hat{x}_ie^{\beta^T\hat {x}_i}}{1+e^{\beta^T\hat{x}_i}})\\
    &=\sum_{i=1}^m((1-y_i)\hat {x}_i-\frac{\hat{x}_i}{1+e^{\beta^T\hat{x}_i}})\\
    \frac{\partial^2 l}{\partial \beta\partial \beta^T}&=
    \sum_{i=1}^m(\hat{x_i}^2(1+e^{\beta^T\hat{x}_i})^{-2}e^{\beta^T\hat{x}_i})
\end{aligned}
$$
显然Hessian是正定的,故对数似然函数函数是凸的。

\answer[3.6]
对所有样本进行一个非线性变换,映射到高维空间,使得映射后的样本满足线性可分的条件。

\answer[3.7]
由于ECOC二元码在长度为9的情况下,最大海明距离为9,则满足最优性的编码中,
类与类之间\textbf{同时}能够满足的最大的海明距离为6,编码策略为:

\begin{itemize}
    \item 110110110
    \item 101101101
    \item 011011011
    \item 000000000
\end{itemize}

下面给出构造过程:考虑码长为3,类别为4的情况,其最优意义下(即每条编码的距离尽量远)的编码为:000,011,101,110。容易验证
最小的海明距离最大为2。将此种编码重复三次,就得到了码长为9的情况。容易验证,上面所设计的编码方案,任意两个ECOC编码的海明距离为6,这意味着没有任何可能找到更优解。(若存在,则需要修改数字0或者1,但上面所提
编码方案若再进行修改,只会引起最小的海明距离的最大值降低。)

\answer[3.9]
因为对于OvR和MvM来说,不同类的处理方法相同,由于对称性,不平衡性产生的影响
可以相互抵消。

\end{document}