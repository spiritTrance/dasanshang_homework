\documentclass{ctexart}
% 此处引入常用包,从此行到46行均无需修改
\usepackage[dvipsnames, svgnames, x11names]{xcolor}
\usepackage{listings}
\usepackage{graphicx}
\usepackage{tabularx}
\usepackage[most]{tcolorbox}
\usepackage{amsmath}
\usepackage{multicol}
\usepackage{multirow}
\usepackage{pifont}
\usepackage{enumitem}
\usepackage{bbding}
\usepackage{colortbl}
\usepackage{placeins}
\usepackage{mathpazo}
\usepackage{bm}
\usepackage{tikz}
\usepackage{xparse}
\usepackage{fancyhdr}
\usepackage[ruled,linesnumbered]{algorithm2e}
\usepackage{algorithmic}

%代码环境设置
\lstset{
    numbers = none ,                                    %可选参数有none,right,left
    breaklines ,                                        %换行有影响,不加这个则换行时从头开始
    numberstyle = \tiny ,                               %数字大小’
    keywordstyle = \color{blue!70} ,                    %关键字颜色
    commentstyle =\color{black!40!white} ,              %注释颜色
    frame = shadowbox ,                                 %阴影设置
    rulesepcolor = \color{red!20!green!20!blue!20} ,    %阴影颜色设置
    escapeinside =`',                                   %lst中文支持不太好,可以用这个括在中文旁边
    basicstyle =\footnotesize\ttfamily                  %代码字体设置
}

%定义题目计数器和命令
\newcounter{questioncnt}
\newcounter{subquestioncnt}[questioncnt]
\newcounter{subsubquestioncnt}[subquestioncnt]

\NewDocumentCommand\question{om}{\noindent\IfNoValueTF{#1}{\textcolor{blue}{\stepcounter{questioncnt}\arabic{questioncnt}}}{#1}\quad#2\par}
\NewDocumentCommand\subquestion{om}{\noindent\IfNoValueTF{#1}{\textcolor{blue}{\stepcounter{subquestioncnt}\arabic{questioncnt}.\arabic{subquestioncnt}}}{#1}\quad#2\par}
\NewDocumentCommand\subsubquestion{om}{\noindent\IfNoValueTF{#1}{\textcolor{blue}{\stepcounter{subsubquestioncnt}\arabic{questioncnt}.\arabic{subquestioncnt}.\arabic{subsubquestioncnt}}}{#1}\quad#2\par}

%定义回答计数器和命令
\newcounter{answercnt}
\newcounter{subanswercnt}[answercnt]
\newcounter{subsubanswercnt}[subanswercnt]

\NewDocumentCommand\answer{o}{\noindent\textcolor{blue}{\IfNoValueTF{#1}{\stepcounter{answercnt}\arabic{answercnt}}{#1}}\quad}
\NewDocumentCommand\subanswer{o}{\noindent\textcolor{blue}{\IfNoValueTF{#1}{\stepcounter{subanswercnt}\arabic{answercnt}.\arabic{subanswercnt}}{#1}}\quad}
\NewDocumentCommand\subsubanswer{o}{\noindent\textcolor{blue}{\IfNoValueTF{#1}{\stepcounter{subsubanswercnt}\arabic{answercnt}.\arabic{subanswercnt}.\arabic{subsubanswercnt}}{#1}}\quad}

%在此处进行基本信息修改
\newcommand{\sCourse}{机器学习}   %课程名
\newcommand{\nTime}{6}             %作业次数
\newcommand{\sName}{黄昊}           %学生姓名
\newcommand{\sNumber}{20204205}     %学号

%页边距设置
\usepackage[left=2cm,right=2cm,top=3cm,bottom=2cm]{geometry}

%页眉页脚设置
\pagestyle{fancy}
\fancyhead[C]{\today}

\newcommand{\homeworkTitle}{
    \setcounter{answercnt}{0}
    %标题部分修改
    \begin{center}
        \fontsize{16pt}{0}{\textbf{\kaishu\sCourse课程\quad第\nTime次作业}}\\
        \fontsize{13pt}{0}{\textit{\kaishu\sName\qquad\sNumber}}\\
    \end{center}}

\begin{document}
    \setcounter{answercnt}{0}
    %标题部分修改
    \homeworkTitle
    选择题目
    \answer[6.1]
    \answer[6.4]
    \answer[6.6]
    \answer[6.7]
    \answer[6.9]

    \answer[6.1]
    显然超平面${w^Tx}+b=0$的法向量为$\boldsymbol{w}$,考虑超平面上一点$\boldsymbol{x_0}$及空间
    中任意一点$\boldsymbol{x}$,有向量$\boldsymbol{x-x_0}$,考虑该向量与法向量的夹角$\theta$,可求
    得点$\boldsymbol{x}$到超平面的距离$\boldsymbol{r}$为:
    $$
    \begin{aligned}
r &= \parallel (\boldsymbol{x-x_0})\parallel _2cos\theta\\
  &= \parallel (\boldsymbol{x-x_0})\parallel _2\frac{|\boldsymbol{w^T(x-x_0)}|}{\boldsymbol{\parallel w\parallel _2\parallel x-x_0\parallel _2}}\\
  &= \frac{|\boldsymbol{w^Tx}+b-(\boldsymbol{w^Tx_0} +b)|}{\parallel \boldsymbol{w}\parallel _2}\\
  &= \frac{|\boldsymbol{w^Tx}+b|}{\parallel \boldsymbol{w}\parallel _2}
    \end{aligned}
    $$

    \answer[6.4]
考虑同一个线性可分的二分类问题,当LDA的投影直线的$\boldsymbol{w}_1$与SVM的超平面$\boldsymbol{w}_2$垂直时,且满足各自的约束条件时,
两者等价,即:
$$
\begin{aligned}
    LDA&:\boldsymbol{w_1}=\boldsymbol{S_w^{-1}(\mu_0-\mu_1)}\\
    SVM&:\boldsymbol{w_2}=\sum_{i=1}^{m}\alpha_iy_i\boldsymbol{x_i}\\
    s.t.&\quad\boldsymbol{w_1^Tw_2}=0
\end{aligned}
$$

    \answer[6.6]
直观上看,SVM的最终模型仅与少数的支持向量有关。此时若噪声成为了最终的支持向量,将会严重影响模型,甚至会
导致线性不可分。因此,SVM对噪声敏感。

    \answer[6.7]
    KKT条件为:
$$
\begin{cases}
    \alpha_i(f(\boldsymbol{x}_i)-y-\epsilon -\xi_i)=0\\
    \hat{\alpha}_i(y_i-f(\boldsymbol{x}_i)-\epsilon-\hat{\xi}_i)=0\\
    \mu_i\xi_i=0\\
    \hat{\mu}_i\hat{\xi}_i=0\\
\end{cases}
$$

    \answer[6.9]
原始的对率回归模型为:
$$
ln\frac{y}{1-y} = \boldsymbol{w^Tx}+b
$$
现在引入核函数,则核对率回归的模型如下所示:
$$
ln\frac{y}{1-y} = \boldsymbol{w^T\phi (x)}+b
$$
不妨令$\beta^T = (w^T, b)$,$\hat{x}=(x^T , 1)$,则模型变为:
$$
ln\frac{y}{1-y} = \boldsymbol{\beta^T\phi (\hat{x})}
$$
相应地,极大似然函数变为:
$$
l(\beta) = \sum_{i=1}^{m}(-y_i\beta^T\phi{\hat{\boldsymbol{x}}}+ln(1+e^{\boldsymbol{\beta}^T\phi \hat{\boldsymbol{x}}}))
$$
由表示定理,有:
$$
ln\frac{y}{1-y}=h(\hat(\boldsymbol{x}))=\sum_{i=1}^m\alpha_i\kappa(\hat{\boldsymbol{x}},\hat{\boldsymbol{x}_i}) 
$$
因此,损失函数变为:
$$
l(\alpha)=\sum_{i=1}^m(-y_i\sum_{j=1}^m\alpha_j\kappa(\hat{x}_i,\hat{x}_j))
+\sum_{i=1}^m ln(1+e^{\sum_{j=1}^m\alpha_j\kappa(\hat{x_i},\hat{x_j})})
$$
对参数求导:
$$
\frac{\partial l}{\partial \alpha_j}=-\sum_{i=1}^m y_i\kappa(\hat{x_i},\hat{x_j})
+\sum_{i=1}^m\frac{\kappa(\hat{x_i},\hat{x_j})}{1+e^{\sum_{j=1}^m \alpha_j\kappa(\hat{x_i},\hat{x_j})}}
e^{\sum_{j=1}^m \alpha_j\kappa(\hat{x_i},\hat{x_j})}
$$
按照该公式即可实现更新。
\end{document}
