\documentclass{ctexart}
% 此处引入常用包,从此行到46行均无需修改
\usepackage[dvipsnames, svgnames, x11names]{xcolor}
\usepackage{listings}
\usepackage{graphicx}
\usepackage{tabularx}
\usepackage[most]{tcolorbox}
\usepackage{amsmath}
\usepackage{multicol}
\usepackage{multirow}
\usepackage{pifont}
\usepackage{enumitem}
\usepackage{bbding}
\usepackage{colortbl}
\usepackage{placeins}
\usepackage{mathpazo}
\usepackage{bm}
\usepackage{tikz}
\usepackage{xparse}
\usepackage{fancyhdr}


%定义题目计数器和命令
\newcounter{questioncnt}
\newcounter{subquestioncnt}[questioncnt]
\newcounter{subsubquestioncnt}[subquestioncnt]

\NewDocumentCommand\question{om}{\noindent\IfNoValueTF{#1}{\textcolor{blue}{\stepcounter{questioncnt}\arabic{questioncnt}}}{#1}\quad#2\par}
\NewDocumentCommand\subquestion{om}{\noindent\IfNoValueTF{#1}{\textcolor{blue}{\stepcounter{subquestioncnt}\arabic{questioncnt}.\arabic{subquestioncnt}}}{#1}\quad#2\par}
\NewDocumentCommand\subsubquestion{om}{\noindent\IfNoValueTF{#1}{\textcolor{blue}{\stepcounter{subsubquestioncnt}\arabic{questioncnt}.\arabic{subquestioncnt}.\arabic{subsubquestioncnt}}}{#1}\quad#2\par}

%定义回答计数器和命令
\newcounter{answercnt}
\newcounter{subanswercnt}[answercnt]
\newcounter{subsubanswercnt}[subanswercnt]

\NewDocumentCommand\answer{o}{\noindent\textcolor{blue}{\IfNoValueTF{#1}{\stepcounter{answercnt}\arabic{answercnt}}{#1}}\quad}
\NewDocumentCommand\subanswer{o}{\noindent\textcolor{blue}{\IfNoValueTF{#1}{\stepcounter{subanswercnt}\arabic{answercnt}.\arabic{subanswercnt}}{#1}}\quad}
\NewDocumentCommand\subsubanswer{o}{\noindent\textcolor{blue}{\IfNoValueTF{#1}{\stepcounter{subsubanswercnt}\arabic{answercnt}.\arabic{subanswercnt}.\arabic{subsubanswercnt}}{#1}}\quad}

%在此处进行基本信息修改
\newcommand{\sCourse}{机器学习}   %课程名
\newcommand{\nTime}{2}             %作业次数
\newcommand{\sName}{黄昊}           %学生姓名
\newcommand{\sNumber}{20204205}     %学号

%页边距设置
\usepackage[left=2cm,right=2cm,top=3cm,bottom=2cm]{geometry}

%页眉页脚设置
\pagestyle{fancy}
\fancyhead[C]{\today}

\newcommand{\homeworkTitle}{
    \setcounter{answercnt}{0}
    %标题部分修改
    \begin{center}
        \fontsize{16pt}{0}{\textbf{\kaishu\sCourse课程\quad第\nTime次作业}}\\
        \fontsize{13pt}{0}{\textit{\kaishu\sName\qquad\sNumber}}\\
    \end{center}}

\begin{document}
    \homeworkTitle
    选择题目\answer[2.1]\answer[2.3]\answer[2.5]\answer[2.7]\answer[2.8]\answer[2.9]

    \answer[2.1]
    正、反例各抽70\%的样本,那么共有$C_{500}^{150}C_{500}^{150}$种划分方式。

    \answer[2.3]
    由公式(2.10)和$BEP$的定义可知,两者无直接关系。因为$BEP$规定了$BEP=P=R$,但$F1$对P和
    R没有具体规定大小关系,不可直接比较。

    \answer[2.5]
    考虑式(2.20)与式(2.21)

    \answer[2.7]

    \answer[2.8]
    min-max规范化的\textbf{优点}是计算简单,当有新元素加入时,只要不是最值,均可以实现O(1)的
    计算;适合在线处理;\textbf{缺点}在于受最值影响大,若样本含有离群点且未作处理将会使其他正
    常值规范化后所得到的值不合理;

    z-score规范化的\textbf{优点}在于鲁棒性强,受极端值影响小;\textbf{缺点}在于每次新加入点
    后都需要重新计算,复杂度为O(n),只适合离线处理。

    \answer[2.9]
    卡方检验在比较两个学习器的性能时的步骤如下:

    首先对符号做如下约定:
    \begin{table}[htbp]
        \centering
        \begin{tabular}{|c|c|c|}
            \hline
            \multirow{2}{*}{算法B}&\multicolumn{2}{c|}{算法A}\\
            \cline{2-3}
            &正确&错误\\
            \hline
            正确&$\epsilon_{00}$&$\epsilon_{01}$\\
            \hline
            错误&$\epsilon_{10}$&$\epsilon_{11}$\\
            \hline
        \end{tabular}
    \end{table}

    使用如下统计量:
    \begin{equation*}
        \tau_{\chi^2}=\frac{(|\epsilon_{01}-\epsilon_{10}|-1)^2}{\epsilon_{10}+\epsilon_{01}}
    \end{equation*}
    
    该统计量服从自由度为1的卡方分布。给定显著度$\alpha$,当上述变量小于该值则不能拒绝假设,
    认为两个学习器的效果没有显著差异;反之则拒绝,认为有明显差异。
    
\end{document}