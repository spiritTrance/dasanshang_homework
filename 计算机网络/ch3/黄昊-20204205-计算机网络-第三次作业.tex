\documentclass{ctexart}
% 此处引入常用包,从此行到46行均无需修改
\usepackage[dvipsnames, svgnames, x11names]{xcolor}
\usepackage{listings}
\usepackage{graphicx}
\usepackage{tabularx}
\usepackage[most]{tcolorbox}
\usepackage{amsmath}
\usepackage{multicol}
\usepackage{multirow}
\usepackage{pifont}
\usepackage{enumitem}
\usepackage{bbding}
\usepackage{colortbl}
\usepackage{placeins}
\usepackage{mathpazo}
\usepackage{bm}
\usepackage{tikz}
\usepackage{xparse}
\usepackage{fancyhdr}
\usepackage[ruled,linesnumbered]{algorithm2e}
\usepackage{algorithmic}


%定义题目计数器和命令
\newcounter{questioncnt}
\newcounter{subquestioncnt}[questioncnt]
\newcounter{subsubquestioncnt}[subquestioncnt]

\NewDocumentCommand\question{om}{\noindent\IfNoValueTF{#1}{\textcolor{blue}{\stepcounter{questioncnt}\arabic{questioncnt}}}{#1}\quad#2\par}
\NewDocumentCommand\subquestion{om}{\noindent\IfNoValueTF{#1}{\textcolor{blue}{\stepcounter{subquestioncnt}\arabic{questioncnt}.\arabic{subquestioncnt}}}{#1}\quad#2\par}
\NewDocumentCommand\subsubquestion{om}{\noindent\IfNoValueTF{#1}{\textcolor{blue}{\stepcounter{subsubquestioncnt}\arabic{questioncnt}.\arabic{subquestioncnt}.\arabic{subsubquestioncnt}}}{#1}\quad#2\par}

%定义回答计数器和命令
\newcounter{answercnt}
\newcounter{subanswercnt}[answercnt]
\newcounter{subsubanswercnt}[subanswercnt]

\NewDocumentCommand\answer{o}{\noindent\textcolor{blue}{\IfNoValueTF{#1}{\stepcounter{answercnt}\arabic{answercnt}}{#1}}\quad}
\NewDocumentCommand\subanswer{o}{\noindent\textcolor{blue}{\IfNoValueTF{#1}{\stepcounter{subanswercnt}\arabic{answercnt}.\arabic{subanswercnt}}{#1}}\quad}
\NewDocumentCommand\subsubanswer{o}{\noindent\textcolor{blue}{\IfNoValueTF{#1}{\stepcounter{subsubanswercnt}\arabic{answercnt}.\arabic{subanswercnt}.\arabic{subsubanswercnt}}{#1}}\quad}

%在此处进行基本信息修改
\newcommand{\sCourse}{计算机网络}   %课程名
\newcommand{\nTime}{3}             %作业次数
\newcommand{\sName}{黄昊}           %学生姓名
\newcommand{\sNumber}{20204205}     %学号

%页边距设置
\usepackage[left=2cm,right=2cm,top=3cm,bottom=2cm]{geometry}

%页眉页脚设置
\pagestyle{fancy}
\fancyhead[C]{\today}

\newcommand{\homeworkTitle}{
    \setcounter{answercnt}{0}
    %标题部分修改
    \begin{center}
        \fontsize{16pt}{0}{\textbf{\kaishu\sCourse课程\quad第\nTime次作业}}\\
        \fontsize{13pt}{0}{\textit{\kaishu\sName\qquad\sNumber}}\\
    \end{center}}

\begin{document}
\homeworkTitle
\answer[3-01]
物理链路仅仅是建立了物理线路段,而数据链路还要加上实现控制数据传输协议的硬件
软件。所以电路接通了仅限于建立起了物理信道,并不能保证点到点的无差错传输;数据
链路接通以后,才能实现无差错传输。

\answer[3-04]
首先数据链路层的职责是保证点对点的无差错传输,而帧定界规定了数据帧的起止位置,如果不加以
规定,接收方就无法从一串比特流中正确地提取数据帧;透明传输需要解决的问题是防止帧定界符的
混淆,如果不加以解决,接收方可能无法正确地确定帧的起止位置;差错检测本身就是为无差错传输
服务的。因为传输过程中可能会有差错,需要数据链路层做差错检测。

\answer[3-07]
经过计算,原数据的FCS为1110,一个1变0的原始数据对应的FCS为1101,两个1变0的原始数据对应的
FCS为1011,因此接收端均能发现。
由于CRC检验到错误时只丢弃帧,没有重传等机制,因此不能实现可靠传输。

\answer[3-09]
7E FE 27 7D 7D 65 7E

\answer[3-10]
经过零比特填充变为:011011111011111000

接收方删除后的比特串:00011101111111111110

\answer[3-13]
主要特点:网络为一个单位所拥有;地理范围和站点数目有限。局域网没有单独设立网络层,
局域网内共享信道,所以采用广播通信方式。而广域网的地理范围相当大,采用广播通信会造成
通信资源的极大浪费。

\answer[3-16]
因为以太网采用曼彻斯特编码,其发送的每一位均有两个信号周期,因此为20M 波特。

\answer[3-20]
单程传播时延为5e-6s,则争用期为1e-5s,因此最短帧长为10000 bits.

\answer[3-24]
10Mb/s的以太网网段要求的最小帧为64字节(51.2$\mu s\times 10 M Bit/s=512 bits=64Bytes$),
加上MAC帧前面加上的8字节同步码,故所花时间为576 比特时间,因为传播时延为225比特,故在225比特
时间时,B站就能收到A站发来的
信号。所以只有前225比特时间内能产生碰撞,若产生碰撞,那么在450比特时间时,A站就能检测
出碰撞。而A站直到发送完毕也没有检测到碰撞,说明一定没有发生碰撞。

\answer[3-25]
首先确定争用期为512比特时间,单程传播时延为255比特时间。基本退避时间为512比特时间,那么
A在255+48+0*510=273比特时间时重传数据,B在255+48+1*512=785比特时间重传数据,那么A传送
的数据帧在273+255=528比特时间时候被B站收到,273+512+8=793比特时间时A站结束传输,由于B在783比特时间
发送数据,此时A站还在传输数据,则B站需要再次采取二进制截断退避算法进行重传。

\answer[3-30]
1100Mbits/s.因为交换机的特点是每个用户独享带宽,而不像集线器那样共享带宽。

\answer[3-33]
忽略有效时间,规定交换表的每一项由二元组$<MAC\mbox{地址,接口}>$组成
\begin{table}[htbp]
    \centering
    \begin{tabular}{|c|c|c|c|}
        \hline
        动作&交换表的状态&向哪些接口转发帧&说明\\
        \hline
        A发送帧给D&加入$<A,1>$&2,3,4,5,6&目的地址在转发表中没有,向所有接口转发\\
        \hline
        D发送帧给A&加入$<D,4>$&1&目的地址在转发表中找得到,故只转发一个接口\\
        \hline
        E发送帧给A&加入$<E,5>$&1&目的地址在转发表中找得到,故只转发一个接口\\
        \hline
        A发送帧给E&不更新&5&目的地址在转发表中找得到,故只转发一个接口\\
        \hline
    \end{tabular}
\end{table}

\end{document}