\documentclass{ctexart}
% 此处引入常用包,从此行到46行均无需修改
\usepackage[dvipsnames, svgnames, x11names]{xcolor}
\usepackage{listings}
\usepackage{graphicx}
\usepackage{tabularx}
\usepackage[most]{tcolorbox}
\usepackage{amsmath}
\usepackage{multicol}
\usepackage{pifont}
\usepackage{enumitem}
\usepackage{bbding}
\usepackage{colortbl}
\usepackage{placeins}
\usepackage{mathpazo}
\usepackage{bm}
\usepackage{tikz}
\usepackage{xparse}
\usepackage{fancyhdr}


%定义题目计数器和命令
\newcounter{questioncnt}
\newcounter{subquestioncnt}[questioncnt]
\newcounter{subsubquestioncnt}[subquestioncnt]

\NewDocumentCommand\question{om}{\noindent\IfNoValueTF{#1}{\textcolor{blue}{\stepcounter{questioncnt}\arabic{questioncnt}}}{#1}\quad#2\par}
\NewDocumentCommand\subquestion{om}{\noindent\IfNoValueTF{#1}{\textcolor{blue}{\stepcounter{subquestioncnt}\arabic{questioncnt}.\arabic{subquestioncnt}}}{#1}\quad#2\par}
\NewDocumentCommand\subsubquestion{om}{\noindent\IfNoValueTF{#1}{\textcolor{blue}{\stepcounter{subsubquestioncnt}\arabic{questioncnt}.\arabic{subquestioncnt}.\arabic{subsubquestioncnt}}}{#1}\quad#2\par}

%定义回答计数器和命令
\newcounter{answercnt}
\newcounter{subanswercnt}[answercnt]
\newcounter{subsubanswercnt}[subanswercnt]

\NewDocumentCommand\answer{o}{\noindent\textcolor{blue}{\IfNoValueTF{#1}{\stepcounter{answercnt}\arabic{answercnt}}{#1}}\quad}
\NewDocumentCommand\subanswer{o}{\noindent\textcolor{blue}{\IfNoValueTF{#1}{\stepcounter{subanswercnt}\arabic{answercnt}.\arabic{subanswercnt}}{#1}}\quad}
\NewDocumentCommand\subsubanswer{o}{\noindent\textcolor{blue}{\IfNoValueTF{#1}{\stepcounter{subsubanswercnt}\arabic{answercnt}.\arabic{subanswercnt}.\arabic{subsubanswercnt}}{#1}}\quad}

%在此处进行基本信息修改
\newcommand{\sCourse}{计算机网络}   %课程名
\newcommand{\nTime}{1}             %作业次数
\newcommand{\sName}{黄昊}           %学生姓名
\newcommand{\sNumber}{20204205}     %学号

%页边距设置
\usepackage[left=2cm,right=2cm,top=3cm,bottom=2cm]{geometry}

%页眉页脚设置
\pagestyle{fancy}
\fancyhead[C]{\today}

\begin{document}
    \setcounter{answercnt}{0}
    %标题部分修改
    \begin{center}
        \fontsize{16pt}{0}{\textbf{\kaishu\sCourse课程\quad第\nTime次作业}}\\
        \fontsize{13pt}{0}{\textit{\kaishu\sName\qquad\sNumber}}\\
    \end{center}


\question{根据你目前的理解,请你构想一个虚拟的网络协议,并将它用你认为恰当的方式描述出来。}

\begin{itemize}
    \item 语法:有三种数据结构,一个是请求连接的数据结构,里面包括发送地址,接收地址,以及请求参数;一个是确认连接,包括请求的发送地址,接收地址,以及表示是否同意的参数,最后一个是数据,包括id,发送地址,接收地址以及数据字段。
    \item 语义:连接发起方首先发送请求连接的控制信息,接收方返回确认连接的控制信息,发起方接收到后开始按id字段的顺序依次发送数据给接收方。
    \item 同步:首先发送端发出请求连接的控制信息,此时有两种情况:一种是收到了确认连接的控制信息,之后开始传输数据;另外一种是没有收到,等待一段时间后重新发送请求连接的控制信息,
    重复以上步骤。传输数据时,将数据切分成数据报,并依次赋一个id,按顺序发送出去。
\end{itemize}

\question{计算机网络体系结构分层次有什么好处?如果采用本教材的五层模型,你认为它们每个层次的主要任务是什么?}

\textbf{第一问}:分层的好处在于各层之间相互独立,不同层完成不同的功能,下层屏蔽掉细节后为上层提供服务,
易于实现和维护。分层后在结构上可以相互分开,灵活性好。

\textbf{第二问}:
\begin{itemize}
    \item 应用层:负责应用进程间的交互。
    \item 运输层:为两台主机的进程间通信提供通用的数据传输服务。
    \item 网络层:负责两台主机间的可靠通信。主要职能为路由选择与转发。
    \item 数字链路层:负责相邻节点之间的可靠通信。
    \item 物理层:负责比特传输。
\end{itemize}

\question{互联网数据传输一般采用什么交换方式?它的特点是什么?}
分组交换。
特点是采用存储转发技术,即每个分组到达路由器后,首先缓存起来,等待路由器查找转发表,
然后将分组进行转发。

\question{根据你目前掌握的知识应该从哪些指标来评价一个网络的性能?}
速率,带宽,吞吐量,时延,时延带宽积,往返时间,利用率。

\end{document}