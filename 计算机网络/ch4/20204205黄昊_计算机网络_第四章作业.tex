\documentclass{ctexart}
% 此处引入常用包,从此行到46行均无需修改
\usepackage[dvipsnames, svgnames, x11names]{xcolor}
\usepackage{listings}
\usepackage{graphicx}
\usepackage{tabularx}
\usepackage[most]{tcolorbox}
\usepackage{amsmath}
\usepackage{multicol}
\usepackage{multirow}
\usepackage{pifont}
\usepackage{enumitem}
\usepackage{bbding}
\usepackage{colortbl}
\usepackage{placeins}
\usepackage{mathpazo}
\usepackage{bm}
\usepackage{tikz}
\usepackage{xparse}
\usepackage{fancyhdr}
\usepackage[ruled,linesnumbered]{algorithm2e}
\usepackage{algorithmic}
\usepackage{array}

%代码环境设置
\lstset{
    numbers = none ,                                    %可选参数有none,right,left
    breaklines ,                                        %换行有影响,不加这个则换行时从头开始
    numberstyle = \tiny ,                               %数字大小’
    keywordstyle = \color{blue!70} ,                    %关键字颜色
    commentstyle =\color{black!40!white} ,              %注释颜色
    frame = shadowbox ,                                 %阴影设置
    rulesepcolor = \color{red!20!green!20!blue!20} ,    %阴影颜色设置
    escapeinside =`',                                   %lst中文支持不太好,可以用这个括在中文旁边
    basicstyle =\footnotesize\ttfamily                  %代码字体设置
}

%定义题目计数器和命令
\newcounter{questioncnt}
\newcounter{subquestioncnt}[questioncnt]
\newcounter{subsubquestioncnt}[subquestioncnt]

\NewDocumentCommand\question{om}{\noindent\IfNoValueTF{#1}{\textcolor{blue}{\stepcounter{questioncnt}\arabic{questioncnt}}}{#1}\quad#2\par}
\NewDocumentCommand\subquestion{om}{\noindent\IfNoValueTF{#1}{\textcolor{blue}{\stepcounter{subquestioncnt}\arabic{questioncnt}.\arabic{subquestioncnt}}}{#1}\quad#2\par}
\NewDocumentCommand\subsubquestion{om}{\noindent\IfNoValueTF{#1}{\textcolor{blue}{\stepcounter{subsubquestioncnt}\arabic{questioncnt}.\arabic{subquestioncnt}.\arabic{subsubquestioncnt}}}{#1}\quad#2\par}

%定义回答计数器和命令
\newcounter{answercnt}
\newcounter{subanswercnt}[answercnt]
\newcounter{subsubanswercnt}[subanswercnt]

\NewDocumentCommand\answer{o}{\noindent\textcolor{blue}{\IfNoValueTF{#1}{\stepcounter{answercnt}\arabic{answercnt}}{#1}}\quad}
\NewDocumentCommand\subanswer{o}{\noindent\textcolor{blue}{\IfNoValueTF{#1}{\stepcounter{subanswercnt}\arabic{answercnt}.\arabic{subanswercnt}}{#1}}\quad}
\NewDocumentCommand\subsubanswer{o}{\noindent\textcolor{blue}{\IfNoValueTF{#1}{\stepcounter{subsubanswercnt}\arabic{answercnt}.\arabic{subanswercnt}.\arabic{subsubanswercnt}}{#1}}\quad}

%在此处进行基本信息修改
\newcommand{\sCourse}{计算机网络}   %课程名
\newcommand{\nTime}{4}             %作业次数
\newcommand{\sName}{黄昊}           %学生姓名
\newcommand{\sNumber}{20204205}     %学号

%页边距设置
\usepackage[left=2cm,right=2cm,top=3cm,bottom=2cm]{geometry}

%页眉页脚设置
\pagestyle{fancy}
\fancyhead[C]{\today}

\newcommand{\homeworkTitle}{
    \setcounter{answercnt}{0}
    %标题部分修改
    \begin{center}
        \fontsize{16pt}{0}{\textbf{\kaishu\sCourse课程\quad第\nTime次作业}}\\
        \fontsize{13pt}{0}{\textit{\kaishu\sName\qquad\sNumber}}\\
    \end{center}}

\begin{document}
    \setcounter{answercnt}{0}
    %标题部分修改
    \begin{center}
        \fontsize{16pt}{0}{\textbf{\kaishu\sCourse课程\quad第\nTime次作业}}\\
        \fontsize{13pt}{0}{\textit{\kaishu\sName\qquad\sNumber}}
    \end{center}
    
    \answer[4-01]
    虚电路服务和数据报服务。
    \begin{table}[htbp]
        \centering
        \begin{tabular}{m{4cm}m{6cm}m{6cm}}
            \hline
            服务种类&优点&缺点\\
            \hline
            虚电路服务&数据能以发送顺序到达终点&结点出故障时,通过该节点的虚电路均不能工作,受到的影响更大\\
            \hline
            数据报服务&路由更灵活,当某些结点出故障时能灵活选择路线&数据不一定以发送顺序到达终点 \\
            \hline
        \end{tabular}
    \end{table}

    \answer[4-03]
    转发器在物理层使用,网桥在数据链路层使用,路由器在网络层使用,网关在网络层以上层使用

    \answer[4-07]
    IP地址在网络层使用,物理地址在数据链路层和物理层使用。
    使用IP地址的意义是屏蔽掉互连的各具体的网络异构细节。而MAC地址是作为用户的唯一标识,具有唯一性。

    \answer[4-10]
    考虑A类地址最高位为0,因此范围为0.0.0.0-127.255.255.255,(2)(5)为A类。


    \quad B类地址最高为10,范围为128.0.0.0-191.255.255.255(1)(3)为B类。

    \quad C类地址最高为110,范围为192.0.0.0-223.255.255.255(4)(6)为C类。


    \answer[4-11]
    好处是可以更快地转发分组,网络层转发更快。缺点是到达终点后,送到运输层的TCP才能发现有无差错,并决定如何处理。

    \answer[4-15]
    MTU为数据链路层规定的传送数据大小的最大值。由于IP层的数据封装后需要送到数据链路层,故IP 数据报首部的总长度字段有关。

    \answer[4-18]
    (1)因为ARP协议需要用到ip地址,所以在网络层。数据链路层不使用IP地址。

    \quad(2)时间太短导致ARP协议的通信过于频繁,占用网络带宽资源;时间太长会导致不易发现网卡的更换。

    \answer[4-24]
    取1的个数:$\lceil\log_2(n+2) \rceil$,其中全0和全1一般不用,所以需要加2。
    因此:

    (1)$\lceil\log_2(2+2) \rceil=2$$\rightarrow$(11111111\quad11000000)$_2\rightarrow$255.192.0.0

    (2)$\lceil\log_2(6+2) \rceil=3$$\rightarrow$(11111111\quad11100000)$_2\rightarrow$255.224.0.0

    (3)$\lceil\log_2(30+2) \rceil=5$$\rightarrow$(11111111\quad11111000)$_2\rightarrow$255.248.0.0

    (4)$\lceil\log_2(62+2) \rceil=6$$\rightarrow$(11111111\quad11111100)$_2\rightarrow$255.252.0.0

    (5)$\lceil\log_2(122+2) \rceil=7$$\rightarrow$(11111111\quad11111110)$_2\rightarrow$255.254.0.0

    (6)$\lceil\log_2(250+2) \rceil=8$$\rightarrow$(11111111\quad11111111)$_2\rightarrow$255.255.0.0


    \answer[4-25]
    (1)176.0.0.0$\rightarrow (10110000...)_2$

    \quad(2)96.0.0.0$\rightarrow (01100000...)_2$

    \quad(3)127.192.0.0$\rightarrow (01111111...)_2$

    \quad(4)255.128.0.0$\rightarrow (11111111\quad 10000000...)_2$

    \quad从二进制写法可以看出,(1)-(3)的子网掩码不连续,只有(4)是连续的。所以推荐使用4。
    
    \answer[4-31]
    $32\rightarrow 00100000$,前四位是0010,匹配的为32-63,96-127,160-191,224-255.
    对照下来,只有(1)匹配

    \answer[4-35]
    $84\rightarrow 01010100$ 取前四位,则最小地址为140.120.80.0/20,最大地址为140.120.95.255/20

    \answer[4-38]
    IGP是内部网关协议,EGP是外部网关协议。内外部的划分依据是看是不是处于一个自治系统。

    \answer[4-39]
    \begin{itemize}
        \item RIP
        \item OSPF
        \item BGP
    \end{itemize}

    \answer[4-43]


    \answer[4-48]


    \answer[4-65]


\end{document}
