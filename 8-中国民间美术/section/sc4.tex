\section{浅谈中国民间美术的出路和发展}
快节奏时代下,中国民间美术确实会因为各种各样的原因受到阻碍。但与此同时,
互联网的发展也十分迅猛,其给中国民间美术带来挑战的同时,也同样带去了机遇。
中国民间美术在快节奏时代下,可以尝试依靠互联网的优势在重重困境中突围。
\subsection{借助互联网的优势进行传播普及}
随着经济发展和科技进步,人们的生活逐渐进入“互联网+”的时代。中国民间美术也可以搭载
“互联网+”的顺风车,与“互联网+”进行有机结合。\upcite{b4}如借助新媒体技术,搭建
数字化民间艺术展馆,打破时空限制,使得更多人能够参与到与民间美术的交互中;又如相关艺术
创作者和从业者可以借助于直播,vlog等方式,记录艺术创作过程,向大众普及各种民间艺术形式,
带领大众欣赏民间艺术之美,构建良好的民间美术氛围。
\subsection{通过文创的形式进行推广普及}
当代年轻人酷爱流行的、现代化的、新奇的事物。而中国民间美术可以结合时下新奇的事物,进行
推广创作\upcite{b5}。如故宫的一系列文创产品就是相当成功的案例。其推出的一系列产品,如口红,折扇等,
借助于故宫的IP,为大众所喜闻乐见。在满足消费者的实际需要时,也能兼具文化性,达到民族文化
传播的目的,从而引导更多人去关注中国民间美术,感受中国民间美术之美。

\subsection{高校发挥教育职能,对学生进行美育教化}
“纯粹之美育,所以陶养吾人之感情,使有高尚纯洁之习惯,而使人我之见、利已损人之思念,以渐消沮者也。”这是一代大师蔡元培对
于美育的评价。作为一所高校,教化学生应成为高校的首要职能。而美育在教学中,
不仅能培养学生的审美观,而且在各种美术作品的背后,往往蕴藏着创作者美好的愿望与积极的思想。潜移默化中,
便如蔡元培大师所述,“陶养吾人之感情,使有高尚纯洁之习惯”,达到教化育人的真正目的。

因此,对于高校,一是要积极开设美育相关的课程,让富有专业知识的老师带领同学欣赏中国
民间美术,解读美术作品后所寄寓的丰富含义,在培养审美的同时,感受中国几千年来的优秀传统文化,
从而实现育人的目的。二是要积极组织开展相关活动,使得更多同学有机会能参与其中,体会中国
民间美术之美;三是要组织建设社团,使得对中国民间美术感兴趣的同学有机会能在一起,相互交流
和沟通,并分派指导老师进行指导,为中国民间美术的传承创造充分的条件。

\zihao{-4}