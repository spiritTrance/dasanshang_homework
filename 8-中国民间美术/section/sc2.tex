\section{浅谈中国民间美术的重要性}
\subsection{中华民族的民族基因}
经过
了几千年的历史演变与变迁,民间美术逐渐成为老百姓必不可少的组成部分,在具有审美价值的
同时,还具有审美和教化功能。譬如,中国剪纸在逢年过节时常贴在家中,既具有装饰功能,其丰富
复杂的图案又蕴含着丰富的含义。如剪纸中常出现的蝙蝠,常成为幸福的象征。而蝙蝠又常以五个的数量
出现,这蕴含着人民对幸福的基本观念——一曰寿,二曰富,三曰康宁,四曰好德,五曰考终命。这是
中国古代劳动人民对于幸福的定义。不仅如此,很多民间美术形式,如剪纸、农民画、刺绣、编织、印染、
风筝等艺术,其实物上的各种图案取材于民间传说、历史典故、传统小说等,具有丰富含义的同时,也使得这
些民间美术形式成为中华民族的象征。

中国民间美术作为中华优秀传统文化的重要组成部分,其承载着中华民族的价值观、审美观。在几千年的
传承中逐渐稳定下来,成为了中华民族的民族基因,为中华民族赓续传承。

\subsection{潜移默化的教育作用}
中国民间美术的教化作用也是不可忽视的。民间美术并不是阳春白雪的高雅艺术,相反,
它是以广大劳动人民为创作者的,接近于大众的一种艺术。
而正由于其为广大人民为创作者的一种艺术,在几千年的历史积淀里,民间美术早已具有
陶冶性情的深厚力量。
鲁迅曾言:“美术之目的,虽与道德不尽
符,然其力足以渊邃人之性情,崇高人之好尚。
亦可辅道德以为治。物质文明,日益蔓延,人情
因亦日趣于肤浅,邪秽之念之作,不待惩劝,而
国又安。”
\upcite{b6}
民间美术往往蕴含着创作者的生活态度和积极向上的情感,在带来审美价值的同时,也在潜移默化地
影响观赏者的心智和品性,发挥其教化的作用。

\subsection{文化自信的必然要求}
习近平总书记在庆祝中国共产党成立95周年大会上提出“四个自信”,即中国特色社会主义道路自信、理论自信、
制度自信、文化自信。而道路自信、理论自信、制度自信,说到底还是要坚持文化自信。文化自信是更基础、更广泛、更深厚的自信;
文化自信是更基本、更深沉、更持久的力量。而为了加强文化自信,我们需要增强文化认同感。如果一个民族对自己的文化认同感低,那么
文化自信便无从谈起。而中国民间美术作为增进文化认同的重要载体,其忠实地记录了祖辈们对于生活的观察与思考。即使
各种各样的民间美术来自于不同地方,其都有一些相通的主题,如求生、求活、求美,这种主
题背后的本质是人性需求的诗意表达。\upcite{b2}即使具有几千年的历史变迁,某些主题却恒久不变,
时至今日,依旧能引起广大群众的共鸣,如积极向上的生活态度,幽默风趣的主题思想
。民间美术背后蕴含的积极的生活态度、美好期望,使其成为一种文化符号,
成为增进文化认同的载体和促进文化自信形成的力量与源泉。
