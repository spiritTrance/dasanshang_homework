\section{浅谈中国民间美术面临的困境及原因}
中国民间美术在快节奏时代下的现状不容乐观:普及程度低,群众认同程度低,常被视为“没有实际价值的东西”……
中国民间美术落得如此现状的原因是多方面的,在快节奏时代下,这些原因较为突出。
\subsection{快餐式文化的盛行}
随着科技的发展,人们娱乐的方式越来越多,越来越现代化。借助于互联网的优势,
短视频文化逐渐流行起来,各种短视频软件如
抖音、快手等,大量输出快餐式内容,为的只是满足人们的低级趣味,
使人们深陷其中。
在短视频的部分劣质内容的传播下,人们将会更偏向于
观看能够带来短暂快乐的短视频,满足于短视频带来的“一时之快”,而对需要一定时间去欣赏的
民间美术逐渐失去兴致。同时,短视频的低质内容,尤其是宣扬“艺术无用论”
的视频,会进一步使短视频软件的受众更为轻视中国民间艺术,认为其“没有实际价值”。在整个社会
浮躁的风气下,中国民间
美术逐渐被冷落,发展逐渐受阻。

\subsection{民间美术教育得不到重视}
尽管国家政府在大力推行中国民间美术的普及,如戏曲进校园的活动一直在进行。但从实际效果上看
收效甚微。其中很大一部分的原因在于校方和学生的不重视。对于中小学来说,校方重视
的是升学质量和教学效果,而对于成绩考核无直接作用的活动,校方往往是临时准备,应付检查。
而老师大多也抱着无所谓,不重视的态度去教学相关内容。\upcite{b3}很多音乐老师没有相关的基础,
学校亦不组织培训,教学往往不了了之。而对于学生来说,一方面对于学业成绩没有直接影响,不会足够重视,
另一个方面在于学生不知道该如何去欣赏民间美术,更没有去欣赏民间美术的耐心。学校到学生的不重视,
使得中国民间美术很少有人去主动欣赏,民间美术的继承和发展更无从谈起。

\subsection{快节奏时代下的浮躁风气}
随着时代的发展,人们的交流变得更加密切和直接,但与此同时,便捷的交流使得信息流通更
为迅速,从而加速了生活的节奏。人们往往需要花费很多精力和时间去赚钱,以满足
衣食住行的物质需要。然而,有相当一部分人在追求物质生活的过程中,变得逐渐浮躁起来,对于
不能帮助其直接获得物质资源的知识,都会被其冠以“无用”的标签,从而以一种鄙视和轻视的态度去
对待他们。而中国民间美术往往是这样的一种存在,其背后蕴含的教化意义和审美价值,都会被
这部分人忽视。长此以往,中国民间美术的推广将会受到这一批人的阻碍。