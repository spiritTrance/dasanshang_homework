\section{前言}
\zihao{-4}
中国民间美术,其是由中国人民群众创作的,具有丰富民间风俗的,在日常生活中所流行的艺术形式。
其最大的特性就在于其为“民间”的美术,而不是属于某一个小部分精英圈子所欣赏的美术。民间美术的创作者
来自于民间的劳动百姓,与我们印象中的美术不同,由于其创作者大多为古代的劳动百姓,其没
有过多的章法和规矩,常以朴素的创作和简单的手法来反映最为朴实的愿望。如中国剪纸艺术中,
鱼、葫芦、老鼠等事物,经常作为一种符号在剪纸作品中出现。这些事物都有一个共同的特性——多子。
因此这些符号出现在剪纸中,常寓意着人们子孙繁衍的愿望。可以说,中国民间美术是我国广大
人民的审美经验和审美理念的集中体现。它集中反映了古代劳动人民对美的感受,民间美术蕴含着
广大民众的审美观念和愿望。而近年来,国家政府对中国民间美术的保护工作越来越重视。
《关于实施中华优秀传统文化传承发展工程的意见》\upcite{b1}指出:文化是民族的血脉,是人民的精神家园。
文化自信是更基本、更深层、更持久的力量。中华文化独一无二的理念、智慧、气度、神韵,增添了中国人民和
中华民族内心深处的自信和自豪。因此,中国民间美术的现状和发展,应足以引起重视和关注。