\documentclass{ctexart}
\usepackage[margin=1in]{geometry}
\pagestyle{plain}  % 页眉页脚设置
\linespread{1.5}  %设置行间距(单倍行距)
% **第一部分:谈论民间美术的重要性:民族基因,文化自信......**
% **第二部分:分析民间美术处的现状和原因:互联网发展下奶头乐盛行;
%  中小学的民间美术普及得不到落实...**
% **第三部分:分析民间美术如何借助互联网时代的优势得到发展:文创,
% 与现代文化符号的有机结合,借助互联网的优势加强宣传工作...**
\newcommand{\upcite}[1]{\textsuperscript{\textsuperscript{\cite{#1}}}}
\begin{document}
\begin{center}
    \zihao{-2}\textbf{\heiti{浅谈快节奏时代下中国民间美术的出路和发展}}
\end{center}
\begin{center}
    \zihao{4}\songti黄昊 20204205 计算机学院
\end{center}
% \begin{center}
%     \zihao{3}\textbf{\heiti{摘要}}
% \end{center}
\zihao{-4}

\noindent\textbf{摘要}:中国民间美术经过了几千年的发展,已经形成多种民间美术形式,如剪纸、刺绣、编织等,
其中所蕴含的审美观、价值观,足以成为中华民族的民族基因。但在当今的快节奏时代下,
中国民间美术的教化作用被大多数人所忽视,面临着重重困境。本文就中国民间美术介绍了
其重要性,并结合快节奏的时代背景分析中国民间美术面临困境的原因,并就快节奏的时代
背景下,如何促进中国民间美术的出路和发展尝试提出了一些建议。


\noindent\textbf{关键词}:中国民间美术\quad 快节奏时代\quad 发展
\thispagestyle{empty}
% \newpage
% \tableofcontents
% \thispagestyle{empty}
% \newpage
\setcounter{page}{1}
\section{前言}
\zihao{-4}
中国民间美术,其是由中国人民群众创作的,具有丰富民间风俗的,在日常生活中所流行的艺术形式。
其最大的特性就在于其为“民间”的美术,而不是属于某一个小部分精英圈子所欣赏的美术。民间美术的创作者
来自于民间的劳动百姓,与我们印象中的美术不同,由于其创作者大多为古代的劳动百姓,其没
有过多的章法和规矩,常以朴素的创作和简单的手法来反映最为朴实的愿望。如中国剪纸艺术中,
鱼、葫芦、老鼠等事物,经常作为一种符号在剪纸作品中出现。这些事物都有一个共同的特性——多子。
因此这些符号出现在剪纸中,常寓意着人们子孙繁衍的愿望。可以说,中国民间美术是我国广大
人民的审美经验和审美理念的集中体现。它集中反映了古代劳动人民对美的感受,民间美术蕴含着
广大民众的审美观念和愿望。而近年来,国家政府对中国民间美术的保护工作越来越重视。
《关于实施中华优秀传统文化传承发展工程的意见》\upcite{b1}指出:文化是民族的血脉,是人民的精神家园。
文化自信是更基本、更深层、更持久的力量。中华文化独一无二的理念、智慧、气度、神韵,增添了中国人民和
中华民族内心深处的自信和自豪。因此,中国民间美术的现状和发展,应足以引起重视和关注。
\section{浅谈中国民间美术的重要性}
\subsection{中华民族的民族基因}
经过
了几千年的历史演变与变迁,民间美术逐渐成为老百姓必不可少的组成部分,在具有审美价值的
同时,还具有审美和教化功能。譬如,中国剪纸在逢年过节时常贴在家中,既具有装饰功能,其丰富
复杂的图案又蕴含着丰富的含义。如剪纸中常出现的蝙蝠,常成为幸福的象征。而蝙蝠又常以五个的数量
出现,这蕴含着人民对幸福的基本观念——一曰寿,二曰富,三曰康宁,四曰好德,五曰考终命。这是
中国古代劳动人民对于幸福的定义。不仅如此,很多民间美术形式,如剪纸、农民画、刺绣、编织、印染、
风筝等艺术,其实物上的各种图案取材于民间传说、历史典故、传统小说等,具有丰富含义的同时,也使得这
些民间美术形式成为中华民族的象征。

中国民间美术作为中华优秀传统文化的重要组成部分,其承载着中华民族的价值观、审美观。在几千年的
传承中逐渐稳定下来,成为了中华民族的民族基因,为中华民族赓续传承。

\subsection{潜移默化的教育作用}
中国民间美术的教化作用也是不可忽视的。民间美术并不是阳春白雪的高雅艺术,相反,
它是以广大劳动人民为创作者的,接近于大众的一种艺术。
而正由于其为广大人民为创作者的一种艺术,在几千年的历史积淀里,民间美术早已具有
陶冶性情的深厚力量。
鲁迅曾言:“美术之目的,虽与道德不尽
符,然其力足以渊邃人之性情,崇高人之好尚。
亦可辅道德以为治。物质文明,日益蔓延,人情
因亦日趣于肤浅,邪秽之念之作,不待惩劝,而
国又安。”
\upcite{b6}
民间美术往往蕴含着创作者的生活态度和积极向上的情感,在带来审美价值的同时,也在潜移默化地
影响观赏者的心智和品性,发挥其教化的作用。

\subsection{文化自信的必然要求}
习近平总书记在庆祝中国共产党成立95周年大会上提出“四个自信”,即中国特色社会主义道路自信、理论自信、
制度自信、文化自信。而道路自信、理论自信、制度自信,说到底还是要坚持文化自信。文化自信是更基础、更广泛、更深厚的自信;
文化自信是更基本、更深沉、更持久的力量。而为了加强文化自信,我们需要增强文化认同感。如果一个民族对自己的文化认同感低,那么
文化自信便无从谈起。而中国民间美术作为增进文化认同的重要载体,其忠实地记录了祖辈们对于生活的观察与思考。即使
各种各样的民间美术来自于不同地方,其都有一些相通的主题,如求生、求活、求美,这种主
题背后的本质是人性需求的诗意表达。\upcite{b2}即使具有几千年的历史变迁,某些主题却恒久不变,
时至今日,依旧能引起广大群众的共鸣,如积极向上的生活态度,幽默风趣的主题思想
。民间美术背后蕴含的积极的生活态度、美好期望,使其成为一种文化符号,
成为增进文化认同的载体和促进文化自信形成的力量与源泉。

\section{浅谈中国民间美术面临的困境及原因}
中国民间美术在快节奏时代下的现状不容乐观:普及程度低,群众认同程度低,常被视为“没有实际价值的东西”……
中国民间美术落得如此现状的原因是多方面的,在快节奏时代下,这些原因较为突出。
\subsection{快餐式文化的盛行}
随着科技的发展,人们娱乐的方式越来越多,越来越现代化。借助于互联网的优势,
短视频文化逐渐流行起来,各种短视频软件如
抖音、快手等,大量输出快餐式内容,为的只是满足人们的低级趣味,
使人们深陷其中。
在短视频的部分劣质内容的传播下,人们将会更偏向于
观看能够带来短暂快乐的短视频,满足于短视频带来的“一时之快”,而对需要一定时间去欣赏的
民间美术逐渐失去兴致。同时,短视频的低质内容,尤其是宣扬“艺术无用论”
的视频,会进一步使短视频软件的受众更为轻视中国民间艺术,认为其“没有实际价值”。在整个社会
浮躁的风气下,中国民间
美术逐渐被冷落,发展逐渐受阻。

\subsection{民间美术教育得不到重视}
尽管国家政府在大力推行中国民间美术的普及,如戏曲进校园的活动一直在进行。但从实际效果上看
收效甚微。其中很大一部分的原因在于校方和学生的不重视。对于中小学来说,校方重视
的是升学质量和教学效果,而对于成绩考核无直接作用的活动,校方往往是临时准备,应付检查。
而老师大多也抱着无所谓,不重视的态度去教学相关内容。\upcite{b3}很多音乐老师没有相关的基础,
学校亦不组织培训,教学往往不了了之。而对于学生来说,一方面对于学业成绩没有直接影响,不会足够重视,
另一个方面在于学生不知道该如何去欣赏民间美术,更没有去欣赏民间美术的耐心。学校到学生的不重视,
使得中国民间美术很少有人去主动欣赏,民间美术的继承和发展更无从谈起。

\subsection{快节奏时代下的浮躁风气}
随着时代的发展,人们的交流变得更加密切和直接,但与此同时,便捷的交流使得信息流通更
为迅速,从而加速了生活的节奏。人们往往需要花费很多精力和时间去赚钱,以满足
衣食住行的物质需要。然而,有相当一部分人在追求物质生活的过程中,变得逐渐浮躁起来,对于
不能帮助其直接获得物质资源的知识,都会被其冠以“无用”的标签,从而以一种鄙视和轻视的态度去
对待他们。而中国民间美术往往是这样的一种存在,其背后蕴含的教化意义和审美价值,都会被
这部分人忽视。长此以往,中国民间美术的推广将会受到这一批人的阻碍。
\section{浅谈中国民间美术的出路和发展}
快节奏时代下,中国民间美术确实会因为各种各样的原因受到阻碍。但在快节奏时代下,
互联网给中国民间美术带来挑战的同时,也带来了机遇。中国民间美术在快节奏时代下,
完全可以依靠互联网的优势在重重困境中进行突围。
\subsection{借助互联网的优势进行传播普及}
随着经济发展和科技进步,人们的生活逐渐进入“互联网+”的时代。中国民间美术也应搭载
“互联网+”的顺风车,与“互联网+”进行有机结合。\upcite{b4}如借助新媒体技术,搭建
数字化民间艺术展馆,打破时空限制,使得更多人能够参与到与民间美术的交互中;又如相关艺术
创作者和从业者可以借助于直播,vlog等方式,记录艺术创作过程,向大众普及各种民间艺术形式,
带领大众欣赏民间艺术之美,构建良好的民间美术氛围。
\subsection{通过文创的形式进行推广普及}
当代年轻人酷爱流行的、现代化的、新奇的事物。而中国民间美术可以结合时下新奇的事物,进行
推广创作\upcite{b5}。如故宫的一系列文创产品就是相当成功的案例。其推出的一系列产品,如口红,折扇等,
借助于故宫的IP,为大众所喜闻乐见。在满足消费者的实际需要时,也能兼具文化性,达到民族文化
传播的目的,从而引导更多人去关注中国民间美术,感受中国民间美术之美。

\subsection{高校发挥教育职能,对学生进行美育教化}
“纯粹之美育,所以陶养吾人之感情,使有高尚纯洁之习惯,而使人我之见、利已损人之思念,以渐消沮者也。”这是一代大师蔡元培对
于美育的评价。作为一所高校,教化学生应成为高校的首要职能。而美育在教学中,
不仅能培养学生的审美观,而且在各种美术作品的背后,往往蕴藏着创作者美好的愿望与积极的思想。潜移默化中,
便如蔡元培大师所述,“陶养吾人之感情,使有高尚纯洁之习惯”,达到教化育人的真正目的。

因此,对于高校,一是要积极开设美育相关的课程,让富有专业知识的老师带领同学欣赏中国
民间美术,解读美术作品后所寄寓的丰富含义,在培养审美的同时,感受中国几千年来的优秀传统文化,
从而实现育人的目的。二是要积极组织开展相关活动,使得更多同学有机会能参与其中,体会中国
民间美术之美;三是要组织建设社团,使得对中国民间美术感兴趣的同学有机会能在一起,相互交流
和沟通,并分派指导老师进行指导,为中国民间美术的传承创造充分的条件。

\zihao{-4}
\section{结语}
本文从中国民间美术的重要性出发,以中国民间美术的重要性说明为什么需要保护中国民间美术,
为中国民间美术寻求出路和发展;然后结合快节奏的时代背景,说明当代社会中,中国民间美术
所面临的困境,分析中国民间美术面临如此困境的原因。最后同样结合快节奏时代的特点,就如何推
动中国民间美术的发展尝试提出了一些建议。相信通过本文的论述,民间美术的从业者和创作者
可以从中得到一些启发,为中国民间美术的传承和发展找到新的出路。

\begin{thebibliography}{00}
    \bibitem{b1} 关于实施中华优秀传统文化传承发展工程的意见[EB/OL]. [2022/11/13]. http://www.gov.cn/gongbao/content/2017/content\_5171322.htm.
    \bibitem{b6} 张纵,王丽莉.当代美术人格教化作用的思考[J].南京艺术学院学报(美术与设计版),2004(03):26-31.
    \bibitem{b2} 王宏飞.增进文化认同与文化自信:中国民间美术的当代价值[J].学习月刊,2021(01):47-48.
    \bibitem{b3} 彭兰. 戏曲进校园在黄冈市中小学的实施调查研究[D].黄冈师范学院,2018.
    \bibitem{b4} 石野飞.“互联网+”背景下民间美术传承方式研究[J].今古文创,2020(37):41-42.
    \bibitem{b5} 许海燕.基于民间美术的文创产品设计创新研究[J].大众文艺,2022(19):44-46.
\end{thebibliography}

\end{document}